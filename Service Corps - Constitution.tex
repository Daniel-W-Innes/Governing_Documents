\documentclass{Service_Corps_Document}
\stdSetup
\begin{document}
\def \Title {Constitution}
\def \Company {Service Corps}
\def \versionNumber {4.0}
\stdFooter
\begin{titlepage}
	\stdTitlePage
\end{titlepage}

\tableofcontents	
		
\newpage
\section{NAME \& ESTABLISHMENT}
\begin{enumerate}
	\item The name of the Company/Crew shall be "Service Corps" established under the Charter of Scouts Canada on September 2011. Currently under the sponsorship of the Royal Canadian Legion Branch 480.
	\item The Company/Crew is part of and subscribes to the By-laws, Policies, and Procedures (BP\&P) of Scouts Canada.
	\item The members of the Company/Crew will endeavor to live according to the Mission, Principles, Practices, and Promise of Scouts Canada. The members of the Company/Crew will endeavor to plan activities (service-oriented, program or social) which will enable all members to participate actively.
\end{enumerate}	
\section{MEMBERSHIP}
\subsection{Applicants}
\begin{enumerate}
	\item Membership of the Company/Crew shall be open to anyone meeting the age criteria established by the BP\&P of Scouts Canada and are willing to live up to the Company/Crew promise, the Aims and Principals of Scouts Canada, and accept the Constitution of the Company/Crew. 
	\item Applicants for full membership shall undergo a trial period, consisting of two business meetings and one major service-oriented activity or two minor service-oriented activity.
	\item Applicants for membership after shall undergo a interview with at least two members of the executive council. Interviewers can approve, or disapprove\footnote{To deny recruitment the executive council Council must provide reason(s) and need the approval of the Advisers and Group Commissioner.} their application to join.
	\item Applicants for full membership shall make the Scout Promise to the assembled Company/Crew upon investiture. 
\end{enumerate}
\subsection{Responsibilities}
\begin{enumerate}
	\item Members of the Company/Crew must abide by the Company/Crew Constitution and Code of Conduct. Failure to abide by the Constitution and Code of Conduct will be reviewed by the executive council and can result in disciplinary action. 
	\item Members of the Company/Crew must provide valid reason for missing meetings. Failure to do so three times consecutively automatically result in suspension of voting privileges.\footnote{An exception can be made for members who are registered within Service Corp but are at school.}
	\item Members of the Company/Crew must provide notice to the executive council at least two weeks in advance of renouncing there membership.
\end{enumerate}
Any stipulation made to a member of the Company/Crew for compliance will also apply to all members of the Company/Crew. All Service Corps members will be placed in one of three categories.
\subsection{Primary}
Those youth and adult members who are registered with the Company/Crew as their primary group. These members shall complete an official Scouts Canada application for membership and submit to the Group Registrar annually. These members are a active part of the Company/Crew and contribute to the organization or execution of events as well as attend regular meetings. The minimum annual requirements of the primary members are as follows: two major events, five elective events, three fundraiser events, all required training, and 75\% of meetings.
\subsection{Secondary}
Those youth and adult members who are registered with another Scouts Canada group and are registered as a duplicate with the Company/Crew. These members shall complete an official Scouts Canada application for membership, print "DUPLICATE" on the top of the form along with their home group and membership number (if possible) and submit to the Group Registrar annually. These members are a active part of the Company/Crew and contribute to the organization or execution of events as well as attend regular meetings. The minimum annual requirements of the secondary members are as follows: two major events, two elective events, two fundraiser events, required training, and 50\% of meetings. 
\subsection{Honorary}
Those youth and adult members who are not necessarily registered with Scouts Canada. These members are not necessarily active members of the Company/Crew andcontribute to the organization or execution of events as well as attend regular meetings. These members must be voted into the group.
\section{EXECUTIVE}
\begin{enumerate}
	\item All executive and non-executive positions will be elected positions.
	\item Executive positions may only be held by full members who have been registered with the Company/Crew for at least one year.\footnote{An exception can be made based on the current group registration.}
	\item There should be at least one member from the Venturer section, one from the Rover section. \footnotemark[\value{footnote}]
	\item Executive positions within the Company/Crew are President, Vice President, Treasurer, Secretary, Quartermaster, and Contact Adviser.
	\item Non-executive positions within the Company/Crew is Master at Arms and Adviser. 
	\item Executive and non-executive position terms begin on September 1st and end on August 31st.
	\item Annual elections for executive positions are to be held in June.
	\item The executive council, with the approval of the contact Adviser, has the authority to suspend any member of the Company/Crew for any duration or permanently revoked membership, as a disciplinary action.
\end{enumerate}
\subsection{Duties of the President}
\begin{enumerate}
	\item Organize executive council and Company/Crew meetings, and ensure all members are notified.
	\item Provide leadership to the Company/Crew at meetings and activities. 
	\item Chair executive council and Company/Crew meetings.
	\item Ensure that all members and Advisers are kept informed.
	\item Attend monthly Group Committee meetings, representing the Company/Crew, including making presentations and other related tasks.
\end{enumerate}
\subsection{Duties of the Vice-President}
\begin{enumerate}
	\item Act in place of the President when required.
	\item Provide leadership to the Company/Crew at meetings and events. 
	\item Keep absentee members informed of current and upcoming activities.
	\item Execute such tasks as assigned by the President.
\end{enumerate}
\subsection{Duties of the Secretary}
\begin{enumerate}
	\item Maintain a record of attendance and notices of absence for all meetings.
	\item Record proper minutes of executive council and Company/Crew meetings. Provide minutes to all members of the meeting.
	\item Report on all internal communication by the Company/Crew.
	\item Keep an up-to-date membership list with phone numbers and e-mail addresses.
	\item Keep copies of the minutes from Group Committee meetings. 
	\item Keep copies of the Company/Crew Constitution and Code of Conduct.
\end{enumerate}
\subsection{Duties of the Treasurer}
\begin{enumerate}
	\item Maintain up-to-date records of the Company/Crew financial state, supported by documented accounts or revenues and expenses.
	\item Submit Company/Crew financial records for Advisers and executive council for review quarterly.
	\item Submit Company/Crew financial records to the Group Committee for auditing annually in September. 
\end{enumerate}
\subsection{Duties of the Quartermaster}
\begin{enumerate}
	\item Maintain a record of all equipment belonging to the Company/Crew.
	\item Ensure the storage and maintenance of equipment belonging to the Company/Crew.
	\item Provide a list of equipment to members planning events for sign-out and use at said events.
	\item Ensure all equipment is necessary and in good repair. 
\end{enumerate}
\subsection{Duties of the Master at Arms}
\begin{enumerate}
	\item Ensure adherence to Robert's Rules of Order.
	\item Ensure adherence to the Constitution and Code of Conduct.
\end{enumerate}
\subsection{Duties of the Advisers}
\begin{enumerate}
	\item To assist the Company/Crew and its members to achieve their objectives, whether group or individual.
	\item Advisers suggest and advise, but do not order members, excepting that Advisers have veto powers on issues of safety, or any laws/by-laws that come into play.
	\item Activities may be held without Advisers present after notification (with complete details) has been provided to the Advisers, and the Advisers have approved the activities.
	\item The contact Adviser is ultimately responsible to the Group Committee for the actions of the Company/Crew. 
\end{enumerate}
A motion of abandonment of position may be raised after any executive has missed three consecutive meetings without a valid reason or has failed to carry out their duties without a valid reason. If for any reason, the members of the Company/Crew wish to add or remove an Adviser, all changes in the Adviser team must first be approved by the Group Commissioner. 
\section{PROGRAM}
\begin{enumerate}
	\item The central program theme of the Company/Crew shall be Community Service. The Company/Crew may also take part in Social Activities and camping. The majority of Company/Crew program planning, training, and resources will be directed toward these activities. 
	\item The Company/Crew at large will determine by vote, other program activities, and objectives for action by the executive council.
	\item The Company/Crew at large will determine the classification of all events, i.e., mandatory, major, elective, or fundraising.
\end{enumerate}
\section{UNIFORMS \& INSIGNIA}
Full members will wear an appropriate order of dress for the he/she is attending activity. The executive council or activity organizer may prescribe an order of dress for an activity. The sponsor of an activity or Company/Crew sponsor may dictate additional uniform requirements.
\begin{enumerate}
	\item No.1A Ceremonial Dress - Scouts Canada uniform shirt with appropriate crests, red/black Service Corps necker, tan pants and black or brown shoes
	\item No.1C Semi-Ceremonial Dress - Scouts Canada uniform shirt with appropriate crests and red/black Service Corps necker
	\item No.3 Service Dress - May consist of, but is not limited to; Scouts activity wear, special items provided from event organizers.
\end{enumerate}
Weather concerns may also dictate uniform requirements. Group necker should be worn on the outside of jackets when activities take place during cold/wet weather. Please note that clothing should be in good condition and appropriate, and if it is not deemed acceptable by the executive council, the youth or adult member may be asked to change or leave the activity.
\section{AMENDMENTS}
Should the need arise to amend the Constitution or Code of Conduct, the following steps should be followed. 
\begin{enumerate}
	\item All members should be informed seven days before a motion to amend the Constitution or Code of Conduct is put forward.
	\item A motion containing the potential amendments is to be put forward at a group meeting.
	\item If passed, a draft detailing the amendment is to be produced and added to the group records.
	\item All amendments must be added to the Constitution or Code of Conduct before the executive's term ends.
\end{enumerate}
\section{DISSOLUTION}
Should the Company/Crew at any time dissolve or cease to exist: 
\begin{enumerate}
	\item All monies and assets purchased and still held at the date of dissolution or cessation of existence shall revert to the Voyageur Council Service Alliance.
	\item Upon payment of any outstanding debts, all remaining funds and assets shall revert to the Voyageur Council Service Alliance, to be held for no less than two years as a contingency for the re-establishment of a new Company/Crew and then distributed for the benefit of the youth of said Group.
	\item All records of the Company/Crew shall be placed under the jurisdiction of the Voyageur Council Service Alliance, in possession of the Group Administrator to be held for no less than two years as a contingency for the re-establishment of the Company/Crew. 
\end{enumerate}
\section{GLOSSARY}
\begin{itemize}
	\item Activity: A meeting or event in which at which Service Corps is attending.
	\item full member: A primary or secondary of  Service Corps.
\end{itemize}
\end{document}