\documentclass{Service_Corps_Document}
\stdSetup
\begin{document}
\def \Title {Constitution}
\def \Company {Service Corps}
\def \versionNumber {4.0}
\stdFooter
\begin{titlepage}
	\stdTitlePage
\end{titlepage}

\tableofcontents	
		
\newpage
\section{NAME \& ESTABLISHMENT}
\begin{enumerate}
	\item The name of the Company/Crew/Unit shall be "Service Corps" established under the Charter of Scouts Canada dated September 2011. Currently under the sponsorship of the Royal Canadian Legion Branch 480.
	\item The Company/Crew is part of and subscribes to the By-laws, Policies, and Procedures of Scouts Canada. The Service Corps Ranger Unit is part of Service Corps and subscribes to the By-Laws, Policies, and Procedures of Girl Guides of Canada.
	\item The members of the Company/Crew will endeavor to live according to the Mission, Principles, Practices, and Promise of Scouts Canada. The members of the Unit will endeavor to live according to the Girl Guides of Canada, to live by the Guide promise. The members of the Company/Crew/Unit will endeavor to plan activities (service-oriented, program or social) which will enable all members to participate actively.
\end{enumerate}	
\section{MEMBERSHIP}
\begin{enumerate}
	\item Membership of the Company/Crew/Unit shall be open to anyone meeting the age criteria established by the "Bylaw, Policies and Procedures" of Scouts Canada and are willing to live up to the Company/Crew/Unit promise, the Aims and Principals of Scouts Canada and Girl Guides of Canada and accept the Constitution of the Company/Crew/Unit. 
	\item Applicants shall undergo a trial period, consisting of two business meetings and one major service-oriented activity or two minor service-oriented activity, after which time at least two members of the Executive will interview, approve, or disapprove\footnote{To deny recruitment the Executive Council must provide reason(s) and need the approval of the Advisors and Group Commissioner. The Executive Council must work with the recruit to find a more suitable Company/Crew/Unit.} their application to join.
	\item Applicants shall make the Scout/Guide Promise to the assembled Company/Crew/Unit upon investiture. 
	\item Members of the Company/Crew/Unit must abide by the Company/Crew/Unit Code of Conduct. Failure to abide by the Code of Conduct will be reviewed by the Executive and can result in disciplinary action. 
	\item Members of the Company/Crew/Unit must abide by Robert's Rules of Order for all Company/Crew/Unit meetings in all cases which it does not conflict with the Company/Crew/Unit Constitution or Code of Conduct.
	\item Members of the Company/Crew/Unit must report any violation of the Company/Crew/Unit Constitution or Code of Conduct to the Executive in a formal and accountable way.
	\item Members of the Company/Crew/Unit must provide notice to the Executive in a formal and accountable way if they are unable to attend a meeting before the meeting takes place.
	\item Members of the Company/Crew/Unit will be subject to a suspension of voting privileges if they miss three meetings without valid reason. Failure to abide will result in review by the Executive and can result in disciplinary action. An exception to this rule applies to youth members who are registered within Service Corp but are away at school.
	\item The Executive, with the approval of the contact Advisor, has the authority to suspend any member of the Company/Crew/Unit for short durations for violations of the Service Corps Code of Conduct. 
	\item Membership in the Company/Crew/Unit can be reviewed and possibly permanently revoked by Service Corps if any member does not comply with this Constitution, the Code of Conduct, Scouts Canada's "Bylaws, Policies and Procedures", Canadian Law, fails to live up to their Scout/Guide Promise, or any combination thereof.
	\item Any member may resign from the Company/Crew/Unit by providing notice of his/her intentions to the Executive at least two weeks in advance.
\end{enumerate}
	
Any stipulation made to a member of the Company/Crew/Unit for compliance will also apply to all members of the Company/Crew/Unit. 
Scouting/Guiding members who participate in the Company/Crew/Unit activities will be placed in one of three categories. Those youth and adults in the Primary and Secondary categories shall be considered FULL members of the Company/Crew/Unit for all purposes including voting privileges. 
\subsection{Primary}
Those youth and adult members who are registered with the Company/Crew/Unit as their primary group. Youth members are typically of Venturer and Rover age. These members shall complete an official Scouts Canada application for membership and submit to the Group Registrar annually. These youth and adults are active members of the Company/Crew/Unit and contribute to the organization or execution of events as well as attend regular meetings. The minimum annual requirements of the primary members are as follows: two major events (in addition to any required training), five elective events, three fundraiser events, and 75\% of meetings.
\subsection{Secondary}
Those youth and adult members who are registered with another Scouts Canada group and are registered as a duplicate with the Company/Crew/Unit or are registered members of Girl Guides of Canada. These members shall complete an official Scouts Canada application for membership, print "DUPLICATE" on the top of the form along with their home group and membership number (if possible) and submit to the Group Registrar annually.\footnote{Rangers are not required to fill out a Scouts Canada registration form.} These youth and adults are active members of the Company/Crew/Unit and contribute to the organization or execution of events as well as attend regular meetings. The minimum annual requirements of the secondary members are as follows: two major events (in addition to any required training), two elective events, two fundraiser events, and 50\% of meetings. 
\subsection{Auxiliary}
Those youth and adult members who are registered with another Group within Scouts Canada or Girl Guides of Canada and who participate in the Company/Crew/Unit activities to complete their required community service, or as a group activity. If training is being offered for a specific event attendance is mandatory. 
\section{EXECUTIVE}
\begin{enumerate}
	\item All Executive and Non-Executive positions will be elected positions.
	\item Executive positions may only be held by members who have been registered with the Company/Crew/Unit for a period of at least one year.\footnote{An exception to this can be made based on the current group registration.}
	\item There should be at least one member from the Venturer section, one from the Rover section and one from the Ranger section on the Executive.\footnotemark[\value{footnote}]
	\item Executive positions within the Company/Crew/Unit are President, Vice President, Treasurer, Secretary, Quartermaster, and Contact Advisor.
	\item Non-Executive positions within the Company/Crew/Unit is Master at Arms and Advisor. 
	\item Executive and Non-Executive position terms begin on September 1st and end on August 31st.
	\item Annual elections for executive positions are to be held in the month of June.
\end{enumerate}
\subsection{Duties of the President}
\begin{enumerate}
	\item Organize Executive and Company/Crew/Unit meetings, and ensure all members are notified, working in conjunction with the Secretary.
	\item Provide leadership to the Company/Crew/Unit at meetings and activities. 
	\item Chair Executive and Company/Crew/Unit meetings.
	\item Ensure all members and Advisors are kept informed.
	\item Attend monthly Group Committee meetings, representing the Company/Crew/Unit including making presentations and other related tasks.
\end{enumerate}
\subsection{Duties of the Vice-President}
\begin{enumerate}
	\item Act in place of the President when required.
	\item Provide leadership to the Company/Crew/Unit at meetings and events. 
	\item Keep absentee members informed of current and upcoming activities.
	\item Execute such tasks as assigned by the President. 
	\item Ensure adherence to the Constitution and Code of Conduct.
\end{enumerate}
\subsection{Duties of the Secretary}
\begin{enumerate}
	\item Organize Executive and Company/Crew/Unit meetings, and ensure all members are notified, working in conjunction with the President.
	\item Maintain a record of attendance and notices of absence for all meetings.
	\item Record proper minutes of Executive and Company/Crew/Unit meetings. Provide minutes to all members of the meeting.
	\item Report on all internal communication by the Company/Crew/Unit.
	\item Keep up-to-date membership list with phone numbers and e-mail addresses.
	\item Keep copies of the minutes from Group Committee meetings. 
	\item Keep copies of the Company/Crew/Unit Constitution. 
\end{enumerate}
\subsection{Duties of the Treasurer}
\begin{enumerate}
	\item Maintain up-to-date records of the Company/Crew/Unit financial state, supported by documented accounts or revenues and expenses.
	\item Report financial activity and status monthly to the Company/Crew/Unit.
	\item Submit Company/Crew/Unit financial records for Advisor(s) and Executive for review quarterly.
	\item Submit Company/Crew/Unit financial records to Group Committee for auditing annually in September. 
\end{enumerate}
\subsection{Duties of the Quartermaster}
\begin{enumerate}
	\item Maintain a record of all equipment belonging to the Company/Crew/Unit.
	\item Ensure the storage and maintenance of equipment belonging to the Company/Crew/Unit.
	\item Provide a list of equipment to members planning events for sign-out and use at said events.
	\item Ensure all equipment is necessary and in good repair. 
\end{enumerate}
\subsection{Duties of the Master at Arms}
\begin{enumerate}
	\item Ensure adherence to Robert's Rules of Order.
\end{enumerate}
\subsection{Duties of the Advisors}
\begin{enumerate}
	\item To assist the Company/Crew/Unit and its members to achieve their objectives, whether group or individual.
	\item Advisors suggest and advise, but do not order members, excepting that Advisors have veto powers on issues of safety, or any rules/laws/by-laws that come into play.
	\item Special meetings may be held without Advisors present after notification (with complete details) has been provided to the Advisors.
	\item Advisors may permit activities without Advisors presence, after approval of details and after the competence of those attending has been demonstrated. 
	\item The contact Advisor is ultimately responsible to the sponsoring Group Committee for the actions of the Company/Crew/Unit. 
\end{enumerate}
A motion of abandonment of position may be raised after any Executive has missed three consecutive meetings without a valid reason or has failed to carry out their duties without a valid reason.
If for any reason the members of the Company/Crew/Unit wish to add or remove an Advisor, all changes in the Advisor team must first be approved by the Group Commissioner of the Voyageur Council Service Alliance. 
\section{PROGRAM}
\begin{enumerate}
	\item The central program theme of the Company/Crew/Unit shall be Community Service. The Company/Crew/Unit may also take part in Social Activities and camping. The majority of Company/Crew/Unit program planning, training, and resources will be directed toward these activities. 
	\item The Company/Crew/Unit at large will determine by vote, other program activities and objectives for action by the Executive.
	\item The Company/Crew/Unit at large will determine the classification of all events, i.e., mandatory, major, elective, or fundraising.
	\item All motions must be put forward to the group. They must be seconded by another member and require a vote of 50\% plus one. No vote can be accepted without quorum. Quorum is 50\% of voting members plus one.
\end{enumerate}
\section{UNIFORMS \& INSIGNIA }
All members will be expected to wear a uniform to all meetings, events, or activities where it enhances the image of the Company/Crew/ Unit (i.e., formal events, community service or fundraising). The sponsor of activities may dictate some uniform requirements. Treat your uniform with care.
\subsection{Scouts Canada Uniform}
\begin{enumerate}
	\item Full Dress Uniform - Scouts Canada uniform shirt with appropriate crests, red/black Service Corps necker, tan pants and black or brown shoes
	\item Full Uniform - Scouts Canada uniform shirt with appropriate crests and red/black Service Corps necker
	\item Activity Uniform - Scouts Canada t-shirt or Golf shirt and red/black Service Corps necker or group apparel.
	\item Event Wear - May consist of, but is not limited to, Scouts activity wear, special items provided from event organizers or specialized unit wear such as scarves, toques, neckers, and vests.
\end{enumerate}
\subsection{Girl Guides of Canada Uniform}
\begin{enumerate}
	\item Youth Uniform - Girl Guides of Canada Blue uniform shirt with Branch-specific trim, White tie with Branch-specific leaf decoration, and pin tab.
	\item Adult Uniform - Girl Guides of Canada Blue uniform shirt, Pin Tab, Scarf.
\end{enumerate}
Weather concerns may also dictate uniform requirements. Group necker should be worn on the outside of jackets when activities take place during cold/wet weather. Auxiliary members are subject to the same uniform stipulations as the rest of the group, substituting their own group necker instead of the Service Corps necker. 
Please note that clothing should be in good condition and appropriate, and if it is not deemed acceptable by the Executive, the youth or adult member may be asked to change or leave the activity.
\section{AMENDMENTS}
Should the need arise to amend the Constitution or Code of Conduct, the following steps should be followed. 
\begin{enumerate}
	\item All members should be informed seven days before a motion to amend the Constitution or Code of Conduct is put forward.
	\item A motion containing the potential amendments is to be put forward at a group meeting.
	\item If passed, a written page of the amendment is to be produced and attached to the Constitution or Code of Conduct until said time that it can be updated.
	\item All amendments must be added to the Constitution or Code of Conduct before the executive’s term ends.
\end{enumerate}
\section{DISSOLUTION}
Should the Company/Crew/Unit at any time dissolve or cease to exist: 
\begin{enumerate}
	\item All monies and assets purchased and still held at the date of dissolution or cessation of existence shall revert to the Voyageur Council Service Alliance.
	\item Upon payment of any outstanding debts, all remaining funds and assets shall revert to the Voyageur Council Service Alliance, to be held for a period of no less than two years as a contingency for the re-establishment of a new Company/Crew/Unit and then distributed for the benefit of the youth of said Group.
	\item All records of the Company/Crew/Unit shall be placed under the jurisdiction of the Voyageur Council Service Alliance, in possession of the Group Administrator to be held for a period of no less than two years as a contingency for the re-establishment of the Company/Crew/Unit. 
\end{enumerate}
\end{document}