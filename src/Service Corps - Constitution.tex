\documentclass{Service_Corps_Document}
\stdSetup
\begin{document}
    \def \Title {Constitution}
    \def \Company {Service Corps}
    \def \versionNumber {4.1}
    \stdFooter
    \begin{titlepage}
        \stdTitlePage
    \end{titlepage}

    \tableofcontents	
            
    \newpage


    \section{NAME \& ESTABLISHMENT}
    \begin{enumerate}
        \item The name of the Company/Crew/Unit shall be "Service Corps" established under the Charter of Scouts Canada dated September 2011. Currently under the sponsorship of the Royal Canadian Legion Branch 480.
        \item The Company/Crew is part of and subscribes to the By-laws, Policies, and Procedures of Scouts Canada. The Service Corps Ranger Unit is part of Service Corps and subscribes to the By-Laws, Policies, and Procedures of Girl Guides of Canada.
        \item The members of the Company/Crew will endeavor to live according to the Mission, Principles, Practices, and Promise of Scouts Canada. The members of the Unit will endeavor to live according to the Girl Guides of Canada, to live by the Guide promise. The members of the Company/Crew/Unit will endeavor to plan activities (service-oriented, program or social) which will enable all members to participate actively.
    \end{enumerate}


    \section{PROGRAM}
    \begin{enumerate}
        \item The central program theme of the Company/Crew/Unit shall be Community Service. The Company/Crew/Unit may also take part in Social Activities and camping. The majority of Company/Crew/Unit program planning, training, and resources will be directed toward these activities. 
        \item The Company/Crew/Unit at large will determine by vote, other program activities and objectives for action by the Executive.
        \item The Company/Crew/Unit at large will determine the classification of all events, i.e., mandatory, major, elective, or fundraising.
        \item All motions must be put forward to the group. They must be seconded by another member and require a vote of 50\% plus one. No vote can be accepted without quorum. Quorum is 50\% of voting members plus one.
    \end{enumerate}


    \section{UNIFORMS \& INSIGNIA }
    All members will be expected to wear a uniform to all meetings, events, or activities where it enhances the image of the Company/Crew/ Unit (i.e., formal events, community service or fundraising).
    The sponsor of activities may dictate some uniform requirements.
    Treat your uniform with care.
    
    \subsection{Scouts Canada Uniform}
    \begin{enumerate}
        \item Full Dress Uniform - Scouts Canada uniform shirt with appropriate crests, red/black Service Corps necker, tan pants and black or brown shoes
        \item Full Uniform - Scouts Canada uniform shirt with appropriate crests and red/black Service Corps necker
        \item Activity Uniform - Scouts Canada t-shirt or Golf shirt and red/black Service Corps necker or group apparel.
        \item Event Wear - May consist of, but is not limited to, Scouts activity wear, special items provided from event organizers or specialized unit wear such as scarves, toques, neckers, and vests.
    \end{enumerate}
    \subsection{Girl Guides of Canada Uniform}
    \begin{enumerate}
        \item Uniform - Girl Guides of Canada navy blue T-shirt or tunic.
    \end{enumerate}

    Weather concerns may also dictate uniform requirements. Group necker should be worn on the outside of jackets when activities take place during cold/wet weather. Auxiliary members are subject to the same uniform stipulations as the rest of the group, substituting their own group necker instead of the Service Corps necker. 
    Please note that clothing should be in good condition and appropriate, and if it is not deemed acceptable by the Executive, the youth or adult member may be asked to change or leave the activity.

    \section{AMENDMENTS}
    Should the need arise to amend the Constitution or Code of Conduct, the following steps should be followed. 
    \begin{enumerate}
        \item All members should be informed seven days before a motion to amend the Constitution or Code of Conduct is put forward.
        \item A motion containing the potential amendments is to be put forward at a group meeting.
        \item If passed, a written page of the amendment is to be produced and attached to the Constitution or Code of Conduct until said time that it can be updated.
        \item All amendments must be added to the Constitution or Code of Conduct before the executive’s term ends.
    \end{enumerate}


    \section{DISSOLUTION}
    Should the Company/Crew/Unit at any time dissolve or cease to exist: 
    \begin{enumerate}
        \item All monies and assets purchased and still held at the date of dissolution or cessation of existence shall revert to the Voyageur Council Service Alliance.
        \item Upon payment of any outstanding debts, all remaining funds and assets shall revert to the Voyageur Council Service Alliance, to be held for a period of no less than two years as a contingency for the re-establishment of a new Company/Crew/Unit and then distributed for the benefit of the youth of said Group.
        \item All records of the Company/Crew/Unit shall be placed under the jurisdiction of the Voyageur Council Service Alliance, in possession of the Group Administrator to be held for a period of no less than two years as a contingency for the re-establishment of the Company/Crew/Unit. 
    \end{enumerate}
\end{document}